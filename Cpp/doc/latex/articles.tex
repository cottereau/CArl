Original references for the theory are (among others)


\begin{DoxyEnumerate}
\item H. Ben Dhia. Multiscale mechanical problems\+: the Arlequin method, {\itshape Comptes Rendus de l\textquotesingle{}Academie des Sciences -\/ Series I\+I\+B 326} (1998), pp. 899-\/904.
\item H. Ben Dhia, G. Rateau. The Arlequin method as a flexible engineering design tool, {\itshape Int. J. Numer. Meths. Engr.} 62 (2005), pp. 1442-\/1462.
\end{DoxyEnumerate}

A non-\/exhaustive list of papers that make use of the software includes


\begin{DoxyEnumerate}
\item R. Cottereau, D. Clouteau, H. Ben Dhia, C. Zaccardi. {\itshape A stochastic-\/deterministic coupling method for continuum mechanics}, Comp. Meth. Appl. Mech. Engr. 200, pp. 3280-\/3288 (2011).
\item R. Cottereau. {\itshape Numerical strategy for the unbiased homogenization of random materials} , Int. J. Numer. Meth. Engr. 95(1), pp. 71-\/90 (2013).
\item C. Zaccardi, L. Chamoin, R. Cottereau, H. Ben Dhia. {\itshape Error estimation and model adaptation for a stochastic-\/deterministic coupling method based on the Arlequin framework} , Int. J. Numer. Meth. Engr. 96(2), pp. 87-\/109 (2013).
\item Y. Le Guennec, R. Cottereau, D. Clouteau, C. Soize. {\itshape A coupling method for stochastic continuum models at different scales} . Accepted for publication in Prob. Engr. Mech. (2013).
\item T. Milanetto Schlittler, R. Cottereau. {\itshape Fully scalable implementation of a volume coupling scheme for the modeling of polycrystalline materials} . Submited to Comp. Mech. (under review) 
\end{DoxyEnumerate}